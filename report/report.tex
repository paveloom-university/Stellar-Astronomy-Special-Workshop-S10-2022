\documentclass[a4paper, oneside]{article}
\special{pdf:minorversion 6}

\usepackage{geometry}
\geometry{
  textwidth=358.0pt,
  textheight=608.0pt,
  top=90pt,
  left=113pt,
}

\usepackage[english, russian]{babel}

\usepackage{fontspec}
\setmainfont[
  Ligatures=TeX,
  Extension=.otf,
  BoldFont=cmunbx,
  ItalicFont=cmunti,
  BoldItalicFont=cmunbi,
]{cmunrm}
\usepackage{unicode-math}

\usepackage[bookmarks=false]{hyperref}
\hypersetup{pdfstartview={FitH},
            colorlinks=true,
            linkcolor=magenta,
            pdfauthor={Павел Соболев}}

\usepackage[table]{xcolor}
\usepackage{booktabs}
\usepackage{caption}

\usepackage{float}
\usepackage{subcaption}
\usepackage{graphicx}
\graphicspath{ {../plots/} }
\DeclareGraphicsExtensions{.pdf}

\usepackage{sectsty}
\sectionfont{\centering}
\subsubsectionfont{\centering\normalfont\itshape}

\newcommand{\su}{\vspace{-0.5em}}

\setlength{\parindent}{0pt}

\newlength{\imagewidth}
\newlength{\imageheight}
\newcommand{\subgraphics}[1]{
\settowidth{\imagewidth}{\includegraphics[height=\imageheight]{#1}}%
\begin{subfigure}{\imagewidth}%
    \includegraphics[height=\imageheight]{#1}%
\end{subfigure}%
}

\hypersetup{pdftitle={Специальный практикум (10-ый семестр, 2022)}}

\begin{document}

\subsubsection*{Специальный практикум (10-ый семестр, 2022)}
\section*{Применение критерия согласия Пирсона}
\subsubsection*{Руководитель: И. И. Никифоров \hspace{2em} Выполнил: П. Л. Соболев}

\vspace{3em}

\subsection*{Задачи}

\begin{itemize}
  \setlength\itemsep{-0.1em}
  \item Оптимизировать параметры моделей нормального и бимодального распределений металличностей по представленным данным, используя метод наибольшего правдоподобия;
  \item Применить критерий согласия Пирсона к каждой комбинации моделей, размеров ячеек и данных.
\end{itemize}

\subsection*{Ход выполнения и результаты}

Модельные функции:

\su
\begin{equation}
  \varphi_\text{normal}(f; \mu, \sigma) = \frac{1}{\sqrt{2 \pi} \sigma} e^{\displaystyle -\frac{(f - \mu)^2}{2 \sigma^2}},
\end{equation}

\su
\begin{equation}
  \varphi_\text{bimodal}(f; \mu_1, \sigma_1, \mu_2, \sigma_2, c) = \frac{c}{\sqrt{2 \pi} \sigma_1} e^{\displaystyle -\frac{(f - \mu_1)^2}{2 \sigma_1^2}} + \frac{1 - c}{\sqrt{2 \pi} \sigma_2} e^{\displaystyle -\frac{(f - \mu_2)^2}{2 \sigma_2^2}}.
\end{equation}


Для этих двух модельных распределений находим статистику

\su
\begin{equation}
  \chi_q^2 = \sum_{j=1}^{J} \frac{(N_j - N p_j)^2}{N p_j}
\end{equation}

и уровень значимости

\su
\begin{equation}
  \alpha_q = P(X \geqslant \chi_q^2) \;\; \text{при} \;\; X \sim \chi^2(k), \;\; k = J - 1.
\end{equation}

Здесь $ N $ --- объём выборки, $ J $ --- число ячеек (разрядов), $ N_j $ --- число объектов, попавших в ячейку с индексом $ j $, а $ p_j $ --- вероятность попадания случайной величины $ f $ в эту ячейку:

\su
\begin{equation}
  p_j = \int_{\hat{f}_j}^{\hat{f}_{j+1}} \varphi_\text{mod}(f) \, df.
\end{equation}

Для нормального распределения имеем

\su
\begin{equation}
  p_j = \int_{\hat{f}_j}^{\hat{f}_{j+1}} \varphi_\text{normal}(f; \mu, \sigma) \, df = \frac{1}{2} \mathrm{erf}{\left( \frac{f - \mu}{\sqrt{2} \sigma} \right)} \bigg|_{\hat{f}_j}^{\hat{f}_{j+1}},
\end{equation}

где $ \mathrm{erf}(z) $ --- функция ошибок:

\su
\begin{equation}
  \mathrm{erf}(z) = \frac{2}{\sqrt{\pi}} \int_0^z e^{-t^2} \, dt.
\end{equation}

Результаты получены с помощью скрипта, написанного на языке программирования \href{https://julialang.org}{Julia}. Код расположен в GitLab репозитории \href{https://gitlab.com/paveloom-g/university/s10-2022/stellar-astronomy-special-workshop}{Stellar Astronomy Special Workshop S10-2022}. Для воспроизведения результатов следуй инструкциям в файле {\footnotesize \texttt{README.md}}.

\newpage

\input{tables/params.tex}
\input{tables/test, 0.1.tex}
\input{tables/test, 0.2.tex}

\newpage

\captionsetup{justification=centering}

\begin{figure}[H]
  \centering
  \setlength{\imageheight}{6.2cm}
  \subgraphics{all/histogram, 0.1}
  \subgraphics{all/histogram, 0.2}
  \caption{Гистограммы данных из файла \texttt{all.dat} \\ и оптимизированные модельные функции}
\end{figure}

\begin{figure}[H]
  \centering
  \setlength{\imageheight}{6.2cm}
  \subgraphics{wo_p1/histogram, 0.1}
  \subgraphics{wo_p1/histogram, 0.2}
  \caption{Гистограммы данных из файла \texttt{wo\_p1.dat} \\ и оптимизированные модельные функции}
\end{figure}

\newpage

\begin{figure}[H]
  \centering
  \setlength{\imageheight}{6.2cm}
  \subgraphics{wo_p12/histogram, 0.1}
  \subgraphics{wo_p12/histogram, 0.2}
  \caption{Гистограммы данных из файла \texttt{wo\_p12.dat} \\ и оптимизированные модельные функции}
\end{figure}

\begin{figure}[H]
  \centering
  \setlength{\imageheight}{6.2cm}
  \subgraphics{wo_p123/histogram, 0.1}
  \subgraphics{wo_p123/histogram, 0.2}
  \caption{Гистограммы данных из файла \texttt{wo\_p123.dat} \\ и оптимизированные модельные функции}
\end{figure}

Таким образом, нуль-гипотеза об описании данных модельным распределением принимается как статистически значимая ($ \alpha_q \geqslant 0.05 $) во всех комбинациях исходных данных, размеров ячеек и моделей. Однако по значениям $ \alpha_q $ и представленным графикам наглядно видно, что бимодальное распределение лучше описывает исходные данные.

\end{document}
